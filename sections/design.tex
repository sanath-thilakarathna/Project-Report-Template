\chapter{Design}

\section{System Block Diagram}
Figure~\ref{fig:block} shows a typical block-level decomposition.

\begin{figure}[H]
  \centering
  % LaTeX-only sample figure (TikZ)
% Use inside a figure environment: % LaTeX-only sample figure (TikZ)
% Use inside a figure environment: % LaTeX-only sample figure (TikZ)
% Use inside a figure environment: \input{figures/block_diagram}

\begin{tikzpicture}[
  block/.style={draw, rounded corners, align=center, minimum height=10mm, minimum width=28mm},
  arrow/.style={-Latex, thick}
]

\node[block] (vin) {Input\\$V_{in}$\\(10--14 V)};
\node[block, right=18mm of vin] (buck) {Buck\\Converter};
\node[block, right=18mm of buck] (vout) {Regulated\\$V_{out}=5$ V};
\node[block, below=12mm of buck, minimum width=40mm] (load) {Embedded Load\\MCU + Sensors};

\draw[arrow] (vin) -- (buck);
\draw[arrow] (buck) -- (vout);
\draw[arrow] (vout) |- (load);

\end{tikzpicture}


\begin{tikzpicture}[
  block/.style={draw, rounded corners, align=center, minimum height=10mm, minimum width=28mm},
  arrow/.style={-Latex, thick}
]

\node[block] (vin) {Input\\$V_{in}$\\(10--14 V)};
\node[block, right=18mm of vin] (buck) {Buck\\Converter};
\node[block, right=18mm of buck] (vout) {Regulated\\$V_{out}=5$ V};
\node[block, below=12mm of buck, minimum width=40mm] (load) {Embedded Load\\MCU + Sensors};

\draw[arrow] (vin) -- (buck);
\draw[arrow] (buck) -- (vout);
\draw[arrow] (vout) |- (load);

\end{tikzpicture}


\begin{tikzpicture}[
  block/.style={draw, rounded corners, align=center, minimum height=10mm, minimum width=28mm},
  arrow/.style={-Latex, thick}
]

\node[block] (vin) {Input\\$V_{in}$\\(10--14 V)};
\node[block, right=18mm of vin] (buck) {Buck\\Converter};
\node[block, right=18mm of buck] (vout) {Regulated\\$V_{out}=5$ V};
\node[block, below=12mm of buck, minimum width=40mm] (load) {Embedded Load\\MCU + Sensors};

\draw[arrow] (vin) -- (buck);
\draw[arrow] (buck) -- (vout);
\draw[arrow] (vout) |- (load);

\end{tikzpicture}

  \caption{System block diagram (LaTeX-only sample)}
  \label{fig:block}
\end{figure}

\section{Topology Selection}
A buck converter is appropriate when stepping down a higher DC input voltage to a lower DC output voltage with high efficiency \cite{ti_buck_design}.

\section{Key Design Calculations (Example)}
For an ideal buck converter, the duty cycle is approximately:
\begin{equation}
  D \approx \frac{V_{out}}{V_{in}}
\end{equation}

Assuming $V_{in}=\SI{12}{\volt}$ and $V_{out}=\SI{5}{\volt}$, then:
\begin{equation}
  D \approx \frac{5}{12} \approx 0.417
\end{equation}

\section{Component Selection}
Document IC selection rationale (availability, features, protection, switching frequency).

\subsection{Bill of Materials (Excerpt)}
\begin{table}[H]
  \centering
  \caption{BOM excerpt (example format)}
  \label{tab:bom}
  % LaTeX-only sample table include
% Use inside a table environment: % LaTeX-only sample table include
% Use inside a table environment: % LaTeX-only sample table include
% Use inside a table environment: \input{tables/bom_excerpt}

\begin{tabular}{@{}llll@{}}
  \toprule
  \textbf{RefDes} & \textbf{Value/Part} & \textbf{Package} & \textbf{Notes} \\
  \midrule
  U1 & Buck Regulator IC & QFN & Synchronous, \SI{500}{\kilo\hertz} \\
  L1 & \SI{4.7}{\micro\henry} & SMD & Saturation current $>\SI{3}{\ampere}$ \\
  COUT & 2 $\times$ \SI{22}{\micro\farad} & 1206 & Low-ESR ceramic \\
  \bottomrule
\end{tabular}


\begin{tabular}{@{}llll@{}}
  \toprule
  \textbf{RefDes} & \textbf{Value/Part} & \textbf{Package} & \textbf{Notes} \\
  \midrule
  U1 & Buck Regulator IC & QFN & Synchronous, \SI{500}{\kilo\hertz} \\
  L1 & \SI{4.7}{\micro\henry} & SMD & Saturation current $>\SI{3}{\ampere}$ \\
  COUT & 2 $\times$ \SI{22}{\micro\farad} & 1206 & Low-ESR ceramic \\
  \bottomrule
\end{tabular}


\begin{tabular}{@{}llll@{}}
  \toprule
  \textbf{RefDes} & \textbf{Value/Part} & \textbf{Package} & \textbf{Notes} \\
  \midrule
  U1 & Buck Regulator IC & QFN & Synchronous, \SI{500}{\kilo\hertz} \\
  L1 & \SI{4.7}{\micro\henry} & SMD & Saturation current $>\SI{3}{\ampere}$ \\
  COUT & 2 $\times$ \SI{22}{\micro\farad} & 1206 & Low-ESR ceramic \\
  \bottomrule
\end{tabular}

\end{table}

\section{PCB Design Considerations}
Include placement rules, high-current loop minimization, ground strategy, and thermal vias. Reference your layout screenshots or 3D render exports. Thermal performance and copper utilization should be justified with appropriate references when relevant \cite{ieee_thermal_pcb}.
